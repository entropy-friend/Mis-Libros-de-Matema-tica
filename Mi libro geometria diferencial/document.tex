\documentclass[12pt]{book}
\usepackage[utf8]{inputenc}
\usepackage[spanish]{babel}
\usepackage[a4paper, total={7in, 10in}]{geometry}
\usepackage{amsmath}
\usepackage{amssymb}
\usepackage{graphicx}
\newtheorem{eje}{Ejemplo}
\newtheorem{teo}{Teorema}
\newtheorem{obs}{Observación}
\newtheorem{lem}{Lema}
\newtheorem{pro}{Proposición}
\newtheorem{ex}{Ejercicio}
\newtheorem{cor}{Corolario}

\newcommand{\z}{\mathbb{Z}}
\newcommand{\n}{\mathbb{N}}
\newcommand{\re}{\mathbb{R}}
\newcommand{\kk}{\mathbb{K}}
\newcommand{\rg}{\mathcal{R}}
\newcommand{\kg}{\mathcal{K}}
\newcommand{\ag}{\mathcal{A}}
\newcommand{\og}{$\mathcal{O}$}

\newcommand{\jku}{J^k(n,1)}
\newcommand{\jkp}{J^k(n,p)}



\begin{document}
	\title{Geometría Diferencial}	
	\author{Abraham Rojas}
	\date{ }
	
	\maketitle
	\tableofcontents
	
	\part{Basics}
	\chapter{Introducción}

\chapter{Variedades diferenciales}

%comentario sobre estructuras exóticas

%cocientes de variedades
Sea $M$ un espacio topológico Hausdorff y segundo contable.\\

$M$ es una \textbf{variedad topológica de dimensión $n$}  si existe un \textbf{atlas}, que es una colección $\mathfrak{A}$ de pares $(U_i, \varphi_i )_{i\in I}$, donde
\begin{enumerate}
	\item $\{U_i\}$ es un cubrimiento abierto de $M$,
	\item para cada $i\in I$, $\varphi_i$ es un homeomorfismo entre $U_i$ y un abierto de $\re^n $.
\end{enumerate}
Los elementos de un atlas son llamados \textbf{cartas}.\\

$M$ es una \textbf{variedad diferenciable} si, para cada $i,j\in I$, $$\varphi_i \circ \varphi_j^{-1} | _{ \varphi_j(U_i \cap U_j)}  \mbox{ es de clase } C^\infty .$$ 

Un mapa diferenciable es una función continua $f:M \rightarrow N$ entre variedades suaves tal que, para todo par de cartas $(U,\varphi)$ y $(V, \psi)$ de $M$ y $N$, respectivamente, tenemos que $$ \psi \circ  f \circ \varphi ^{-1} |_{ f^{-1}(V) \cap U }  \mbox{ es de clase } C^\infty.$$

\begin{pro}
	Las variedades diferenciables forman una categoría, tomando como morfismos los mapas diferenciables.
\end{pro}

Los isomorfismos en esta categoria son llamados \textbf{difeomorfismos}.

\begin{eje}
	
\end{eje}

Una variedad compleja de dimensión (compleja) $n$ 





\chapter{Espacio tangente}

\section{Campos vectoriales}

%flujos y campos completos




\chapter{Algunas aplicaciones}

\section{Subvariedades}

\section{Grupos y Álgebras de Lie}


\section{Propiedades de las variedades}

\subsection{Orientación}

\subsection{Partición de la unidad}

\subsection{Preludios...}



\chapter{Formas diferenciales}

\section{Derivada (exterior)}

\section{Caso complejo}

\section{El Complejo de De Rham}
%invariancia por homotopia

%teorema de Rham en libro de topologia algebraica



\subsection{La sequencia de Mayer-Vietoris}






\chapter{Integración}

\section{Teorema de Stokes}


\section{Dualidad de Poincaré}


\section{Género}

\section{Teoría de grado en variedades}








	
\part{Geometría Riemanniana y simpléctica}
	
	
	
	
	
\chapter{Variedades tóricas}	


\chapter{Kähler package}


\chapter{Categoría de Fukaya}


	
\part{Clases Características}





	
\end{document}








%cobordismo
%ya debería saber toería de morse 

