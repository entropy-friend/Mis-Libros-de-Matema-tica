\documentclass[12pt]{book}
\usepackage{gfsartemisia-euler}
\usepackage[T1]{fontenc}
\usepackage[utf8]{inputenc}
\usepackage[spanish]{babel}
\usepackage[a4paper, total={7in, 10in}]{geometry}
\usepackage{tgbonum}
\usepackage{amsmath}
\usepackage{amssymb}
\usepackage{graphicx}
\newtheorem{eje}{Ejemplo}
\newtheorem{teo}{Teorema}
\newtheorem{obs}{Observación}
\newtheorem{lem}{Lema}
\newtheorem{pro}{Proposición}
\newtheorem{ex}{Ejercicio}
\newtheorem{cor}{Corolario}
\renewcommand{\familydefault}{\sfdefault}
\newcommand{\z}{\mathbb{Z}}
\newcommand{\n}{\mathbb{N}}
\newcommand{\re}{\mathbb{R}}
\begin{document}
\large	
	\title{Topología Algebraica}	
	\author{Abraham Rojas}
	\date{2023}
	
	\maketitle
	\tableofcontents
	\part{(Co)homologías}

	%estructura del libro 

\chapter{Algunos conceptos topológicos}

El término mapa denota función continua. $I$ denotará el intervalo $[0,1]$. Términos como retracción, cocientes, serán pensados en la categoria de espacio topológicos.

Dado un mapa $f: X \rightarrow Y$, el \textbf{cilindro} de $f$ es el cociente de $ (X\times I )  \bigsqcup Y $, identificando $(x,1)$ con $f(x)$


\section{Homotopías}

Una \textbf{homotopía} entre dos espacio $X$ e $Y$ es una familia de mapas $f_t: X \rightarrow Y$ con $t\in I$, tal que $F: X \times I \rightarrow Y$ definida por $F(x,t ) = f_t (x)$ es continuo. Diremos que $f_0 $ y $f_1$ son homotópicos, y escribiremos $f_0 \simeq f_1$.\\

Un \textbf{retrato por deformación} de $X$ es un subespacio $A$ es una homotopía entre $Id_X$ y una retracción de $X$ en $A$.\\
Una homotopía $f_t : X\rightarrow Y$ que se restringe a la identidad en $A$ es una \textbf{homotopía relativa} a $A$.\\

Un mapa $f: X \rightarrow Y$ es una \textbf{equivalencia homotópica} si existe otro mapa $g: Y \rightarrow X$ tal que $fg \simeq Id$ y  $gf \simeq Id$, en este caso, $X$ e $Y$ son \textbf{homotópicamente equivalentes} (o tiene el mismo \textbf{tipo de homotopía}), denotaremos $X \simeq Y$.\\
Un espacio con la clase de homotopía de un punto es llamado \textbf{contráctil}. Equivalentemente, su mapa identidad es homotópica a un mapa constante (hacia algún punto).
Note que un retrato por deformación es una equivalencia homotópica.

\begin{eje}
\begin{enumerate}
\item La cinta de Möbius se retrae por deformación a su círculo interior. 
\item Consideremos una hoja de papel con dos hoyos, esta se retrae a lo siguiente... pag. 2, por tanto estas figuras son homotópicamente equivalentes, pero no son retratos de deformación entre sí. \textit{Esto puede expresarse en función de cilindros}.
\item Se puede retraer una X gorda en una $X$ fina y luego retraer a un punto. Observe que los caminos pueden cruzarse, a diferencia de los ejemplos anteriores. 
\end{enumerate}
\end{eje}

\begin{ex}
Sea $f: X \rightarrow Y$ un mapa, $M_f$ se retrae por deformación a $Y$, lo Cuál generaliza los ejemplos anteriores. 
\end{ex}


\section{Complejos simpliciales }


\section{Complejos CW}

\begin{eje}
Construir un toro partir de un cuadrado, y generalizar para un polígono regular de $4g$ lados paar obtener superficies orientables.
\end{eje}


son construido de forma inductiva
\begin{enumerate}
\item $X^0$ es un conjunto discreto, sus puntos son las $0-$células del complejo.
\item Inductivamente, construiremos el $n$-esqueleto $X^n$ a partir de $X^{n-1}$, adjuntando $n$-células $e_\alpha^n$ via mapas $\varphi_\alpha: S^{n-1} \rightarrow X^{n-1}$. Esto significa que $X^n$ es el cociente de $X^{n-1} \bigsqcup_\alpha D_\alpha^n$ of $X^{n-1}$, con las identificaciones $x \sim \varphi_\alpha(x)$ para $x \in \partial D_\alpha^n$. Tenemos que $X^n=X^{n-1} \bigsqcup_\alpha e_\alpha^n$ donde cada $e_\alpha^n$ es un $n$-disco abierto.
\item Podemos parar este proceso luego de $n$ pasos, obteniendo un complejo de \textbf{dimensión} $n$, o dejarlo continuar indefinidamente, en cuyo caso $X$ tiene la topología débil: $A \subset X$ es abierto si y solo si $A \cap X^n$ es abierto en $X^n$ para todo $n$.
\end{enumerate}


\begin{eje}
\begin{enumerate}
\item Un \textbf{grafo} es un complejo celular de dimensión $1$.
\item $S^n$ tiene dos células. 

\item \textbf{Espacio proyectivo real de dimensión $n$.} Existen definiciones equivalentes:
\begin{enumerate}
	\item Conjunto de subespacios lineales de dimensión $1$ en $\mathcal{R}^{n+1}$.
\item Cociente de $R^{n+1} \setminus \{0\} $, identificando puntos en una misma recta.
\item Cociente de $S^n$, identificando puntos antípodas.
\item Cociente del hemisferio $ D^n $, identificando puntos antípodas en $\partial D^n$.
\end{enumerate} Como el cociente de $\partial D^n$ identificando antípodas es $\mathcal{R}P^{n-1}$, tenemos una estructura de complejo celular $\mathbb{R}P^n = e^0 \cup e^1 \cup \cdots \cup e^n $, donde $e^i$ es una $i-$célula, $0 \leq i\leq n$. Si no detenemos este proceso, obtenemos $\mathbb{R} \mathrm{P}^{\infty}=\bigcup_n \mathbb{R} \mathrm{P}^n$


\item El caso complejo es similar, tenemos que $\mathbb{C P}^n=e^0 \cup e^2 \cup \cdots \cup e^{2 n}$, y es equivalente al cociente de $S^{2n+1}$ (identificamos no solo antípodas, sino aquellos múltiplos por complejos de norma $1$).
\end{enumerate}
\end{eje}

%terminar cap 0







\chapter{Homologías}

\section{Cálculo}




\section{Aplicaciones clásicas}

\subsection{Teorema de De Rham}


\subsection{Grado}




\section{Axiomas}	


	
	
\section{De los coeficientes}	





\chapter{Cohomología}


\section{Dualidad de Poincaré}

\section{Productos cup y cap}




\part{Teoría de Homotopía}


\chapter{Grupo fundamental}

\section{Cálculos}

\section{Cubrimientos}

\section{Una aplicación en Teoría de Grupos}










\part{Más herramientas}	


\chapter{Homología de intersección}



\chapter{Espacios de intersección}
\end{document}




