\documentclass[12pt]{book}
\usepackage[utf8]{inputenc}
\usepackage[spanish]{babel}
\usepackage[condensed,math]{anttor}
\usepackage[T1]{fontenc}
\usepackage[a4paper, total={7in, 10in}]{geometry}
\usepackage{amsmath}
\usepackage{amssymb}
\usepackage{graphicx}
\newtheorem{eje}{Ejemplo}
\newtheorem{teo}{Teorema}
\newtheorem{obs}{Observación}
\newtheorem{lem}{Lema}
\newtheorem{pro}{Proposición}
\newtheorem{ex}{Ejercicio}
\newtheorem{cor}{Corolario}

\newcommand{\z}{\mathbb{Z}}
\newcommand{\n}{\mathbb{N}}
\newcommand{\re}{\mathbb{R}}
\newcommand{\kk}{\mathbb{K}}
\newcommand{\rg}{\mathcal{R}}
\newcommand{\kg}{\mathcal{K}}
\newcommand{\kp}{(\mathbb{K}^p,0)}
\newcommand{\ag}{\mathcal{A}}
\newcommand{\oo}{\mathcal{O}_n}
\newcommand{\mm}{\mathfrak{m}_n}
\newcommand{\dif}{\operatorname{Dif}}
\newcommand{\dem}{\textbf{Demostración.}}
\newcommand{\cod}{\operatorname{cod}}
\newcommand{\knp}{(\mathbb{K}^n \times \mathbb{K}^p)}
\newcommand{\kn}{(\mathbb{K}^n,0)}
\newcommand{\jku}{J^k(n,1)}
\newcommand{\jkp}{J^k(n,p)}



\begin{document}
	\title{Teoría de singularidades}	
	\author{Abraham Rojas}
	\date{ }
	
	\maketitle
	\tableofcontents
	

	\chapter*{Introducción}
Uno de los problemas fundamentales es clasificar las singularidades de aplicaciones diferenciales u holomorfas, que en adelante llamaremos regulares.\\
Solo nos interesa el comportamiento de la función en una vecindad de la singularidad.\\

Comenzaremos estudiando la estructura algebraica de los gérmenes de funciones. 
	
%yo lo colocaría después de saber alg conmutativa y análisis real (eso viene de graudaución , creo yo)

\section{Resultados de acciones de grupos en variedades}

\begin{teo}
	Si las órbitas son subvariedades, entonces los mapas $\phi_p:G \rightarrow G\cdot p$ es una sumersión. En particulpar $$ LG \cdot p = d \varphi_p (T_1 G).  $$
\end{teo}



\begin{pro}[Clasificación formas cuadráticas con $n=2$]
Sea $f=ax^2 +2bxy + c^2$, el rango coincide con el rango del Hessiano, y del número de variables de la forma normal.
\begin{itemize}
\item \textbf{Simbólico.} $f=0$, rango $0$
\item \textbf{Parabólico.} rango $1$, $b^2=ac$
\item \textbf{Hiperbólico.} rango $2$, semi-índice $1$, $ b^2 -ac>0$.
\item \textbf{Elíptico.} rango $2$, semi-índice $0$, $ b^2 -ac<0$. en el caso complejo, son \textbf{no degenerados}
\end{itemize}
\end{pro}

\begin{pro}[Clasificación formas cúbicas con $n=2$]
Sea $p= l_1 l_2 l_3$, $l_i$ forma lineal sobre $\mathbb{C}$. en $\Re$, todas son reales o una es real y las otras son conjugadas.
\begin{itemize}
\item \textbf{Cúbica Simbólica.} $l_i$ colineales, equivalente a $x^3$
\item \textbf{Parabólica.} $l_1, l_2$ colineales, $l_2,l_3$ no colineales, $x^2 y$.
\item \textbf{Cúbica no degenerada.} en $\mathbb{C}$, $l_i$ nunca colineales, $ x^2 y + y^3  $
\item \textbf{hiperbólica.} En $\Re$, $l_i$ nunca colineales y reales, $x^2 y - y^3$.
\item \textbf{elíptica.} en $\mathbb{R}$, $l_i$ nunca colineales, una real, $ x^2 y + y^3  $.
\end{itemize}
\end{pro}


\begin{teo}[Lema de Mather]
	Sea $G$ un grupo de Lie actuando en una variedad suave de forma .... Sea $N\subset M$ una subvariedad. Tenemos que $N$ está contenida en una órbita si y solo si 
	\begin{enumerate}
		\item $N$ es conexa.
		\item $T_pN \subset LGp$ para todo $p \in N$.
		\item $\dim G.p$ no depende de $p\in N$
	\end{enumerate}
\end{teo}

%los ejemplos anteriores tabmie´n son consecuencia del Lema de Mather !!!!!

\begin{cor}[espacio afines]
	Sea $A$ un espacio afín, $G$ actuando tal que las ´robitas son subvariedades. Sea $W \leqslant V_A$, $x+W$ está contenida en una órbita si \begin{enumerate}
		\item $W \hookrightarrow LGx$ 
		\item $LG (x+ w) \simeq LGx$ para todo $w\in W$.
	\end{enumerate}
\end{cor}


\begin{teo}[Transversal completa]
	Sea $G$ actuando en $A$ afín. Sea $W \leqslant V_A$ tal que $$ LG (x+w) \simeq LGx, \; \forall x \in A , \; w \in W $$
	Entonces
	\begin{enumerate}
		\item Para todo $x\in A$: $$ x+ ( LGx \cap W ) \subset Gx + \{ x+ W\} $$\\
		\item Si $x_0 \in A $ y $T \leqslant W$ tal que $$ W \subset LGx_0 +L $$ entonces $$\forall w \in W , \; \exists g \in G \mbox{ tal que }  g \cdot (x_0 + w) = x-0 +t$$  
	\end{enumerate}
	
\end{teo}

\part{Clasificación de singularidades}

\chapter{Funciones $(\mathbb{K}^n, 0) \rightarrow \mathbb{K}$}

\section{El álgebra de gérmenes}
%notación de gérmenes
%geermenes de conjuntos?

Definimos \begin{eqnarray*}
\mathcal{O} _{n,p} =\{ f:(\mathbb{K}^n,0) \rightarrow \mathbb{K}^p \mbox{ regular}\}\\
\mathcal{O}_{n,p}^{0} = \{f\in \mathcal{O}_{n,p} \;|\; f(0)=0\}.
\end{eqnarray*}
Denotaremos $\mathcal{O}_{n,1}$ por $\mathcal{O}_n$, que es un anillo local, con ideal maximal $ \mathcal{O}_{n,1}^0$, denotado por $\mathfrak{m}_n$ o $\mathfrak{m}$.\\ 
Es claro que $\mathcal{O}_{n,p}$ es un $\oo$-módulo libre de rango $p$.\\


El siguiente resultado permite encontrar generadores para $\mathfrak{m}_n$. Note la semejanza con el caso algebraico, allí eso es una consecuencia inmediata de las definiciones.

\begin{teo}[Lema de Hadamard]
Sea $f: U \times \mathbb{K}^q \rightarrow \mathbb{K}$ suave tal que $f(0,y) =0$ para todo $y\in \mathbb{K}^q$, $U\subset\mathbb{R}^n$ conjunto estrellado . Luego existen funciones $f_1, \ldots f_n: U \times \mathbb{K}^q \rightarrow \mathbb{K}$ tales que $f= x_1 f_1 + \cdots + x_n f_n$, donde $f_i (x,y) =  \int^1 _0 \partial_i f (tx, y) dt  $.
\end{teo}%tiene otras aplicaciones interesantes
\begin{cor}	
Dado $k\geq 1$, tenemos que $\mathfrak{m}_n ^k $ es generado por los monomios de grado $k$ en $n$ variables. 
\end{cor}




Sea $\varphi : \mathcal{O}_n \rightarrow K[[x_1, \ldots x_n]]$ el mapa que asigna las series de Taylor. 
\begin{teo}[Lema de Borel]
	$\varphi$ es un epimorfismo $\mathbb{K}$-álgebras, cuyo núcleo es nulo en el caso complejo, y $\mathfrak{m}_n^\infty$ (funciones planas) en el caso real, que es un ideal no nulo. 
\end{teo}

\begin{cor}
	En el caso real, $\oo$ no es un anillo noetheriano.
\end{cor}
\textbf{Demostración.} Supongamos que $\mm$ sea finitamente generado, como $\mm ^\infty\subset \{0\} + \mm \mm^\infty$ entonces el lema de Nakayama implica que $ \mm \subset \{0\}$, contradiciendo el teorema anterior.
\begin{cor}
Sea $\hat{\mm}^k$ la imagen de $\mm $ por $\phi$, tenemos que $$ \oo / \mm ^k \simeq \hat{\oo} / \hat{\mm}^k \simeq \mathbb{K}[x_1 , \cdots , x_n] _{<k}$$
\end{cor}


El siguiente resultado es importante en determinación finita y otros contextos
\begin{pro}
	Sea $M$ un $\oo-$módulo libre de rango finito y sea $N$ un submódulo. Luego $\dim _\mathbb{K} M/N$ es finita si y solo si existe un entero $k\geq 1$ tal que $\mathfrak{m} ^k M \subset N$.
\end{pro}


%\section{Transversalidad y Teorema de Sard }
%
%\begin{teo}[Lema básico de transversalidad]
%Sea $F:N \times S\rightarrow P $ suave, sea $Q_i$ una familia finita de subvariedades de $P$. Si $F$ transversal a todos los $Q_i$ entonces existe un conunto denso de parámetros tal que $f_s$ es transversal a todos los $Q_i$.
%\end{teo}
%
%%añadir estos tópicos en el libro de variedades diferencialbes





El \textbf{espacio de $k-$jets} $J^k (n,p)$ es el conjunto de $p-$uplas de polinomios en $n$ variables de grado $\leq k$ con término independiente nulo.\\
El espacio de jets posee estructura de variedad suave, de las cual hablaremos más adelante.\\
Sea $f:U \rightarrow \mathbb{K}^p $ suave/holomorfa, definimos su $k-jet$ en $a$, $ j^k f :U \rightarrow J^k (n,1) $, como el polinomio de Taylor de orden $k$ en $a$. 

\begin{cor}
$$\mm ^k = \{f \in \oo \; | \; j^{k-1} f (0) =0\}$$
\end{cor}
%no es trivial y requiere la construcción del lema de haddamard






%\begin{ex}[Nakayama fuerte]
%Sea $I$ un ideal tal que para $x\in I$, $1+x$ es una unidad. Sea $M$ un $A-$módulo, sean $A,B \leqslant M$ tal que $A$ es finitamente generado. Si $A \subset B+ IA$ entonces $A \subset B$. 
%\end{ex}




\section{$\rg-$equivalencia}


%definir codimensión extendida como la dimensión extendida.
Definimos
$$ \rg = \operatorname{Aut} \kk^n (0,0)$$
cuya acción en $\mathcal{O}_n$ es $ h \cdot f = f \circ h^{-1} $.\\
Definición de morfismos inducido... morfismos entre álgebras de funciones.\\

Una familia interesante de gérmenes son los\textbf{ finitamente determinados}, aquellos cuyo $k-$jet determina su $\rg-$clase.

$\rg$ y $\mathcal{O}$ no son variedades suaves. Para solucionar este problemas y también entender la determinación finita, vamos a considerar $$ \rg ^{(k)} = \{ j^k h (0) \; | \; h \in \mathcal{R} \}  = \frac{\rg}{\rg _k}, \mbox{ donde } \rg _k = \{ j^k h(0) = \operatorname{id}_{\mathbb{R}^n} \}  $$
que es un grupo de Lie, y actúa en $ \jku $ por $ j^k h (0) \cdot j^k f(0) = j^k (f \circ h ^{-1}) (0)$.

\begin{pro}
gérmenes equivalentes tienen la misma determinanción.
\end{pro}

Para caracterizar la determinación finita necesitaremos los siguientes resultados, que son importantes por sí solos.


%recordar sobre acciones en general y grupos de Lie
%llamar phi mapa orbital... gneeralziar en mi libro de álgebra––––––




%


%$\{ j^k h(0) \; |\; h \in \mathcal{R} \}$ es un subgrupo normal de $\mathcal{R}$.

%%%%%%

\begin{pro}\label{tanrk}
$$ L \mathcal{R} ^{(k)} h = \frac{\mathfrak{m}_n J f + \mathfrak{m}_n ^{k+1}}{\mathfrak{m}_n ^{k+1}} =\left\{j^{k}\left(\sum_{i=1}^{n} \frac{\partial \sigma}{\partial x_{i}} h_{i}\right) (0)  \; | \; \left(h_{1}, \ldots, h_{n}\right) \in J^{k}(n, n)\right\}  $$
\end{pro}


\begin{lem}[Lema de Thom-Levine]
Sea $f\in \oo$ y $F':(\mathbb{K}^n \times \mathbb{K},0) \rightarrow (\mathbb{K}\times \mathbb{K}, 0) $ tal que $F'(x,t) = (F(x,t),t)$, $F(0,t)=0$, $F (x,0) = f(x)$. Si $\displaystyle\frac{\partial F}{\partial x}  \in  \mm \displaystyle\left\langle \frac{\partial F}{\partial x_1}, \cdots \frac{\partial F}{\partial x_n} \right\rangle \mathcal{O}_{n+1} $ entonces existe $H:(\mathbb{K}^n \times \mathbb{K},0) \rightarrow (\mathbb{K} ^n\times \mathbb{K}, 0)$ tal que $H(x,0) =x$, $H(0,t)=0$ y $F(\cdot, t) \circ H(\cdot, t) = f$ para todo $t$.
\end{lem}



\begin{lem}
las órbitas son subavariedades regulares!
\end{lem}

\begin{teo}[Determinación finita para $\rg$] Sea $f\in \oo$.
\begin{enumerate}
\item Si  $\mm ^{k+1} \subset \mm^2 J(f)$ entonces $f$ es $k-$determinado (basta que $\mm^k \subset \mm J(f)$).
\item Si $f$ es $k-$determinado entonces $\mm^{k+1}\subset \mm J(f)$
\end{enumerate}
\end{teo}

motivación\\

El \textbf{ideal Jacobiano} está definido por $$
J(f)=\left\langle\frac{\partial f}{\partial x_{1}}, \ldots, \frac{\partial f}{\partial x_{n}}\right\rangle
$$

\begin{pro}
\begin{enumerate}
\item  $f\in \mm^2$ si y solo si $Df=0$ si y solo si es crítico si y solo si $J(f) \subset \mm$.
\item $f\in \mm ^2$ es no degenerado si y solo si $Jf =\mm$.
\item $f\in \mm^3$ si y solo si es crítico degenerado
\end{enumerate}
\end{pro}


La Proposición \ref{tanrk} motiva la siguiente definición (tomando $k \rightarrow \infty$). Sea $f\in \mathcal{O}^n$. El \textbf{espacio tangente a $\mathcal{R}f$} está definido como $$
T \mathcal{R} \cdot f= \mathfrak{m}_n J(f)= \left\{\sum_{i=1}^{n} \frac{\partial f}{\partial x_{i}} h_{i} / h_{i} \in \mathfrak{m}_{n}, i=1, \ldots, n\right\}
$$
Definimos la \textbf{codimensión} de $f$ por $\rg-\operatorname{cod} (f) = \dim_\mathbb{K} \mm / \mm J(f)$.
 %La motivación es natural considerando lo anterior.\\
\begin{cor}[Determinación finita y codimensión]
	Un germen en $\oo$ es finitamente determinado si y solo su codimensión es finita.
\end{cor}

Existe otra noción de espacio tangente, la de \textbf{espacio tangente extendido}... motivación. Más adelante daremos otra descripción de él.\\
Definimos la \textbf{codimensión extendida} por $\rg_e\cod f = \dim _{\mathbb{K}} \oo / J(f)$, también llamada \textbf{número de Milnor} también denotadad por$\mu(f)$. Esta será la que usaremos en adelante, cuando hablemos de codimensión.

\begin{pro}
\begin{enumerate}
\item Sea $f\in \mm$, $\rg-\cod f$ es finita si y solo si $\mu(f)$ lo es, si y solo si $\mm^k \subset J f$.
\item Sea $f\in \mm$ tal que $0< \rg-\cod f < \infty$. Entonces $\rg - \cod f = \mu (f) + n -1$.
\item $\mu (f)=0$ si y solo si es no singular, $\mu (f)=1$ si y solo si es crítico no degenerado.
\end{enumerate}
\end{pro}



\begin{ex}
Si $n \geq 2$ y $f\in \mm^2$ es independiente de una de su variables entonces tiene codimensión infinita.
\end{ex}

%LOS ITALIANOS DEFINEN CODIMENSION muy diferente (tampoco es la usual), PERO MARCELO Y OTROS USAN EL NUMERO DE MILNOR

\begin{teo}[Invarianza]
La codimensión es un $\rg-$invariante.
\end{teo}




%Definimos el \textbf{espacio tangente extendido} como $L\mathcal{R}_e f =J(f)$.
%El \textbf{número de Milnor} de $f$ es $\mathcal{R}_e-$cod$(f)$, denotado por $\mu (f)$. En la parte II mostraremos otras interpretaciones de este número, y su importancia.\\
%Por la pro..., tenemso que $\mathbb{R}-\operatorname{cod}(f)$ es finita si y solo si existe $k>0$ tal que $J(f) \subset \mathfrak{m}_n^k$.\\
%
%Conexión entre la acción y la topología de la singularidad.
%
%\begin{teo}
%Sea $f\in \mathfrak{m}_n$ de $\mathbb{R}-\operatorname{cod}(f)$ finita. Entonces es un singularidad aislada.
%\end{teo}
%
%\begin{pro}
%En el caso complejo, vale la recíproca del teorema anterior.
%\end{pro}
%
%\begin{eje}
%contraejemplos
%\end{eje}
%
%\begin{pro}Sea $f \in \mathfrak{m}_n$ tal que $0< \mathcal{R}-\operatorname{codim}(f) < \infty$. Luego
%$$
%\mathcal{R}-\operatorname{codim}(f)=\mathcal{R}_{e}-\operatorname{codim}(f)+n-1
%$$
%\end{pro}










\begin{teo}[Codimensión y Topología]
Si la codimensión es finita entonces la singularidad es aislada. La recíproca vale en el caso complejo.
\end{teo}














\chapter{Las catástrofes elementales}

historia...\\
Thom clasificó las singularidades hasta codimensión $5$, llamadas \textbf{catástrofes elementales}.\\
Vamos clasificar las singularidades según codimensión, posibles valores del corango.\\

%estamos considerando funciones con imagen 0

% no se puede clasficar finitamente para codimensiones mayores

\begin{pro}
$\mu(f)= 0$ si y solo si $f$ es un sumersión. En particular, es equivalente a una proyección.
\end{pro}



\textbf{Clasificación de formas cuadráticas}. El rango es invariante lineal, pero el índcie puede cambiar, para eso definimos el semi-índice. Sale con álgebra lineal o aplicando el Lema de Mather. Cuando $\mathbb{K}= \mathbb{C}$ entonces el índice es igual al rango. \\ % el índice es el número de positivos



Sea $f \in \mm ^2$. Es \textbf{no degenerada} si la matriz hessiana (en cero) es no singular, caso contrario es \textbf{degenerado}. Definimos el \textbf{corango} de $f$ como $\operatorname{cr} f = n- \operatorname{rk} \operatorname{Hess} f $. Definimos el \textbf{índice} de $f$ como el índice de su Hessiana.



\begin{teo}[Lema de Morse]
$\cod f =1$ si y solo si es no degenerada. En ese caso, si $s$ es el índice de $f$, entonces es equivalente a $$ x_1 ^2  + \cdots + x_s ^2 -x _{s+1} - \cdots - x_{n}^2 $$
\end{teo}
% 2 determinado +  clasif formas cuadráticas


\begin{teo}[Lema de Separación]
Si $f$ es degenerada (siempre trabajamos con gérmenes en el origen) de corango $c$ entonces existe $g\in \mathfrak{m}_c ^3$, único por $\mathcal{R}-$equivalencia y $\mu(f)= \mu (g)$ojn     , tal que $f$ es equivalente a $$g(x_1, \ldots, x_c) \pm x_{c+1 }^2 + \ldots + \pm x^2_n.$$
\end{teo}


\begin{pro}[Cota para el corango]
	Sea $f\in \mm ^2$. Sea $k= \operatorname{cor} f$ Entonces $ \mu (f) \geq \frac{1}{2} k (k+1) +1.$
\end{pro}

Por tanto, analizaremos por ahora hasta corango $2$.\\
%esto también opcional




\begin{teo}[Singularidades $A_k$]
Sea $f\in \mm^2$ de corango $1$ y $\mu (f)=k$. Entonces $f$ es equivalente a $$  \pm x^{k+1}  \pm x_2^2 + \pm \cdots \pm x_n^2 .$$
\end{teo}
\dem Separación. Tenemos $g: (\mathbb{K},0) \rightarrow(\mathbb{K},0)$, entonces $J(g)= x^k$... \\
En el caso complejo, $g$ es equivalente a $x^{k+1}$. En el caso real, si $k$ es par entonces $Ak$ es equivalente a $A_{-k}$. Si $K$ es impar, lo contrario.\\

Para el caso corango $2$, haremos inducción sobre el grado de los jets, usando el método de la transversal.\\ %explicar mejor el algoritmo.
Sea $f$, $W = H^{k+1} (n,1) = \frac{\mm^k}{\mm^{k+1}}$, $A = j^k (f) + W \subset J^{k+1}(n+1)$. Así, vamos a buscar $T\leqslant W$ tal que $L \rg f + T = j^k (f) + W$ (note que aquí sí es igualdad). 

\begin{pro}[Transversal completa en $J^{k+1}\rg$]
Sea Sean $G_1, \ldots G_r \in H^{l+1}(n,1)$ que cumplen $$  \mm ^2 J(f) + \mathbb{R} \{G_1 , \ldots , G_r\}  + \mm ^{k+2} \supset \mm ^{k+1} $$ entonces todo $g \in J^{k+1}(n,1)$ con $j^k g = j^k f$ es $\rg_1^{(k+1)}$-equivalente a algún $ f+ \sum _{i=1}^r  u_i G_i $
\end{pro}




\begin{pro}[Singularidades $D_k$]
Si $f\in \mm^3$ de codimensión $k -1 \geq 3$y $j^3 g$ es no degenerada o parabólica. Entonces es equivalente a $$xy^2 \pm y^{k-1} $$
\end{pro}



Falta el caso simbólico, de codimensión mayor que 5.\\
Podemos dar por ahora la clasificación de Thom hasta codimensiçon 5.\\

Antes de continuar, debemos tener en cuenta que una clasificación finita no siempre es posible. Esto envuelve la noción de germen simple, que será explicada en el próximo capítulo.\\




%obviamente, la catastrofes elementales no son gérmenes estables, codimensión no nula...

%ade singularities, conexiones con álgebra de lie sem simples







\chapter{$\ag$ vs. $\kg$ equivalencia}
Dejaremos de enfocarnos en funciones, y consideraremos mapas más generales.\\

Diremos que dos mapas son a equivalente cuando...\\
Definir L equivalencia también\\

Como en el capítulo pasado, nos gustaría definir la codimensión, mostrar que es un invariante, y encontrar criterios de determinación finita. Además, noción de espacio tangente seguirá siendo. Por otro lado, podemos daruna motivación más geométrica que la de truncamientos \\

Imitando la intuición de la teoría de variedades suaves, usaremos despliegues para denotar caminos en el espacio de funciones. Recordemos que la acción está siendo aplicada, en principio, en $\mm$\\
En realidad, solamente lograremos clasificar gérmenes estables, i.e., de codimensión $0$, para $n,p$ pequeños.\\



\section{Despliegues}

Los despliegues generalizan la idea de caminos en el espacio de funciones.\\

Sea $f: (\mathbb{K}^n, 0) \rightarrow (\mathbb{K}^p)$.
Un \textbf{despliegue a $d-$parámetros} es un germen $F':(\mathbb{K}^n \times \mathbb{K}^d, 0) \rightarrow (\mathbb{K}^p, \mathbb{K}^d)$.

\begin{eje}
	teorema de Thom-Levine y otros (germen constante)... espacios tangentes, aunque dejaremos esa construcción para después, nuestra principal motivación fueron las acciones truncadas.
\end{eje}

Dos despliegues a $r-$parámetros $F_1$, $F_2$ de un germen $f: (\mathbb{R}^n)\rightarrow (\mathbb{R}^p)$ son \textbf{equivalentes} si existen despliegues $I_n$, $I_p$ de de los mapas identidad en $\mathbb{R}^n$, $\mathbb{R}^p$, respectivamente, tal que $$ F_2 = I_p ^{-1} \circ F_1 \circ I_{n}$$
es decir, son $\ag-$equivalentes. Note que $I_n$ (y deformaciones asociadas) son gérmenes invertibles, similarmente con $I_p$.\\
Un despliegue es trivial si es equivalente al germen constante. Un germen es \textbf{estable} si todo despliegue es $\mathcal{A}-$trivial.\\




\begin{eje}
	ejemplos de gérmenes estables y no estables
\end{eje}

Despliegue inducido, es como un cambio de parámetros.\\
Despliegue versal, cualquier otro despliegue es equivalente a un despliegue inducido.\\
Despliegue isomorfos si es equivalente a un despliegue inducido por un difeomorfismo.

\begin{teo}[Versalidad y transversalidad]
Un \textbf{despliegue $F$} a $d$-parámetros es versal si y solo si \textbf{transversal}, i.e., $$ L \ag _e f_0 + \langle \stackrel{\cdot}{F}_1, \ldots , \stackrel{\cdot}{F}_d \rangle _ \mathbb{R} = \mathcal{O}_{n,p} $$
o sea, si las derivadas generan el cotangente....
\end{teo}


\begin{cor}[Versalidad y codimensión]
Un germen tiene codimensión finita si y solo si admite un despliegue versal, en ese caso, el número de parámetro de una deformación miniversal es la codimensión extendida.
\end{cor}



\subsection{Conjunto de bifurcación}

El \textbf{conjunto de catástroges} de un despliegue es $$ C(F) = \{ (x,u) \in (\mathbb{K}^n \times \mathbb{K}^d ,0 ) \; | \; DF (x,u) \mbox{ es singular }  \} $$
El \textbf{conjunto de bifurcación} es $$ B(F) = \Sigma (F) = \{ u \in (\mathbb{K^n},0) \; |\; \exists (x,u) \in C(F), \; \operatorname{Hess} (f) \mbox{ es singular } \} $$
el discrimiante es la imagen el conjunto de catástrofes. 
\begin{pro}
despliegues versales con el mismo número de parámetros tienen todo isomorfo
\end{pro}


\subsection{Espacio tangente}


Definimos el \textbf{espacio tangente} a $f$ por $$ L\ag f = \{ \left. \frac{d}{dt} \psi_{t} \circ f \circ \varphi_{t}\right\vert_{ t=0} : \phi_t,\varphi_{t} \mbox{ deformaciones de las identidades tq. } \psi_t (0)=0, \; \varphi_{t} (0)=0 , \; \forall t  \} $$

%sus elementos son los centros de organización de los depliegues derivados (germenes en el origen)

Una definición útil será la de \textbf{espacio tangente extendido} a $f$, dada por $$ L\ag_e f = \{ \left. \frac{d}{dt} \psi_{t} \circ f \circ \varphi_{t}\right\vert_{ t=0} : \phi_t,\varphi_{t} \mbox{ deformaciones de las identidades} \} $$

Para dar una definición equivalente de espacio tangente, que será mucho más calculabe, usaremos las siguientes notaciones
\begin{itemize}
	\item $\theta_n$: gérmenes de campos vectoriales sobre $\mathbb{K}^n$ en $0$, es un $\oo$-módulo de rango $n$.
	\item $\theta(f)$: gérmenes de campos vectoriales a lo largo de $f$, es un $\oo$-módulo libre, y también un $\mathcal{O}_p$-móudulo libre usando $f^*$.
	\item $tf:\theta_n \rightarrow \theta(f)$ dada por $\xi \mapsto df \circ \xi$, morfismo de $\oo$-módulos
	\item $wf : \theta _p \rightarrow \theta(f)$ dada por $ \eta \mapsto \eta \circ f$, morfismo de $\mathcal{O}_p-$módulos
\end{itemize}



\begin{lem}
$$
\left.\frac{d}{d t}\left(\psi_{t} \circ f \circ \varphi_{t}^{-1}\right)\right|_{t=0}=d f \circ\left(\left.\frac{d \varphi_{t}^{-1}}{d t}\right|_{t=0}\right)+\left(\left.\frac{d \psi_{t}}{d t}\right|_{t=0}\right) \circ f
$$
\end{lem}


\begin{teo}[espacio tangente]
$$ L \ag f = tf (\mm\theta_n)  + \omega f (\mathfrak{m}_p \theta_p ) = \mm J(f) + \{ \eta \circ f \;|  \; \eta \in \mathfrak{m}_p  \mathcal{O}_{p,p}\} $$
$$ L \ag_e f = tf (\theta_n)  + \omega f (\theta_p ) = J(f)\oo + \{ \eta \circ f \;|  \; \eta \in \mathcal{O}_{p,p}\} $$
\end{teo}
Note que tenemos una combinación de estructuras modulares.\\
El concepto de codimensión significa cuanto le falta a un espacio para ser igual al total. En nuestro caso, el espacio total será el espacio tangente a $f$ en el espacio de funciones, que identificaremos con el conjunto $ \mm \theta (f) $. 
En el caso extendido, tiene sentido considerar el espacio de todas las funciones, no solo de las que fijan el origen.\\

Por tanto, las codimensiones serán definidas por $$ \ag-\operatorname{cod} (f) = \dim _{\mathbb{K}} \frac{\mm\mathcal{O}_{n,p}}{L\ag f} \quad  \ag_e -\operatorname{cod} (f) = \dim _{\mathbb{K}} \frac{\mathcal{O}_{n,p}}{L\ag_e f} $$




\section{Equivalencia de contacto}

Dados dos gérmenes $f_1, f_2: \kn \rightarrow \kp$, sabemos que sus gráficos son subvariedades de $\mathbb{K}^n\times \mathbb{K}^p$. La $\kg$-equivalencia los relaciona si estos gráficos tienen el mismo \textit{tipo de contacto} en $(0,0)$ con el eje $\mathbb{K}^n$. En este caso, buscamos que estas intersecciones sean isomorfas.\\
la ventaja de la $\kg$-equivalencia es que es mucho más fácil de verificar. Además está relacionada fuertemente con la $\ag-$equivalencia, cuando consideramos relación entre germenes y despliegues, próximo capítulo.\\


El grupo de contacto es definido por 
$$ \mathcal{K} = \{ (h,H) \in \dif \kn \times \dif \knp \;| \; H(x,y) = (h(x), \theta (x,y)), \; \theta (x,0)=0 \} $$
que actúa en $f: \kn \rightarrow \kp$ por $$ (h,H) \cdot f (x)= \theta (h^{-1}(x), f(h^{-1} (x))) $$

%dibujar /  mostrar que $A$.equi es un caso particular

Un caso particular es el grupo $\mathcal{C}$, tomando $h= \operatorname{id}_{\kn}$. 

\begin{pro}
$\kg$ es el producto semi-directo de $\rg$ y $\ag$.
\end{pro}

\begin{pro}[criterio algebraico de equivalencia]
Sean $ f,g \in \mathcal{O}_{n,p} ^0$. Son equivalentes
\begin{enumerate}
\item $f,g$ son $\mathcal{C}-$equivalentes
\item Los ideales $\langle f_1, \ldots, f_p  \rangle = \langle g_1, \ldots, g_p  \rangle$ en $\oo$.
\item Existe una matriz $(u_{ij})\in \oo ^{p\times p} $  tal que $f_i= \sum u_{ij} g_j$. 
\end{enumerate}
\end{pro} 
%demostración en Gibson 146


\begin{eje}
K equi pero no A equi
\end{eje}

\begin{teo}[criterio algebraico de equivalencia]
Dos gérmenes son $\kg$-equivalentes si y solo sus $\langle f_1, \ldots, f_p  \rangle = \langle g_1, \ldots, g_p  \rangle$ son isomorfos inducidos.
\end{teo}

Como antes, podemos definir los espacios tangentes y la codimensiones, $\mathcal{O}_n-$módulo.
$$ L \mathcal{K} f = tf (\mm) + f^* \mathfrak{m}_P\theta(f) = J(f) \mm + I(f) \mathcal{O}_{n,p}  $$
$$ L \mathcal{K}_e f = tf (\oo) + f^* \mathfrak{m}_P\theta(f) = J(f)  + I(f) \mathcal{O}_{n,p}  $$
Todos vistos como submódulos de $\mathcal{O}_{n,p}$
%\begin{pro}
%Si $\mm ^k \theta(f) \subset L\kg f$ para algún $k$ entonces la $\kg-$codimsión es finita.
%\end{pro}

% no encuentro la demostración!! es necesaria?? SI, para la exsitencia de despliegues transversales.



\begin{teo}
La $\kg$-codimensión es $\kg-$invariante.
\end{teo}


%en dimensión finita 

%definir grupos de mather


\begin{teo}
Un despliegue es $\kg$-versal si y solo si es $\kg$-transversal, si y solo si la $\mathcal{K}-$codimensión es finita
\end{teo}

%no sé donde está la demostración
\begin{pro}
Dos deformaciones miniversales son isomorfas.
\end{pro}
\dem mostrar que el cambio de base entre ambos es un difeo.\\


A continuación mostraremos un resultado fácil y fundamental. 

\begin{pro}
	Todo germen es $F:\kn \rightarrow\kp$ de rango $r$ es $\re$-equivalente al despliegue a $r-$parámetros de un germen de rango $0$.  
\end{pro}

%trivial, solo cambio de coordenadas

Por tanto, en la $\ag$-clasificación de gérmenes, podemos restringirnos a despliegues.

\begin{pro}
si son despliegues son $\ag-$equivalentes entonces los centros son $\kg-$equivalentes.
\end{pro}







\chapter{Estabilidad}


Hay varias maneras de entendar la estabilidad: topologicamanete, con despliegues, e infinitesimalmente.

Sea $F': (\mathbb{K}^n \times \times \mathbb{K}^r) \rightarrow (\mathbb{K}^p \times \times \mathbb{K}^r)  $ un despliegue a $r-$parámetros. Vamos anlaizar la correspondencia $F \mapsto F_0$.

\begin{teo}[fundamental]
	Existe una biyección entre las clases de $\ag$-equivalencia de gérmenes estables $(\mathbb{K}^r \times \mathbb{K}^n) \rightarrow (\mathbb{K}^r \times \mathbb{K}^p)$ de rango $r$ y las clases de $\kg-$equivalencia de rango $0$ y $\kg _e$-codimensión $\leq r+p$
\end{teo}

\section{Estabilidad infinitesimal}
Un germen es dicho $\mathcal{G}$-\textbf{infinitesimalmente estable} si la $\mathcal{G}_e$-codimensión es  $0$.
 %comentario en cada capítulo sobre el camion que seguiremos, o sino en la introducción...
\begin{teo}
Estable (por despliegues) es equivalente a infinitesimalmente estable.
\end{teo}


%muy largo, se aplica Thomo Levine





%uno de los resultados envolvidos no está demostrado en Gibson... parece largo.

Así, nos enfocaremos en hallar las clases de $\kg-$equivalencia gérmenes de rango $0$ hasta cierta $\kg_e $codimensión.\\

Podemos descomponer este problema en otros más fáciles utilizando la \textit{estratificación} (partición) de Thom-Boardman, que utiliza la topología del espacio de funciones, que finalmente comenzaremos a utilizar, en lugar de huir de ella.


%esto sirve para probar estabilidad de gérmenes, supongo!!
%toerema de Sard, transversalidad de Thom


\section{La Topología de Whitney}

Nos gustaría que el conjunto de aplicaciones estables sea abierto y denso. Seremos globales por un rato.\\

Escoja una métrica $d$ en $J^k(n,p)$ (que es una variedad suave). La \textbf{topología $C^k$ de Whitney} tiene como base $$ V(f, \delta) = \{ g\in C^k \; |\; d (j^k g (x) - j^k f(x) ) < \delta (x), \; \forall x\in \mathbb{K} \} $$
donde $\delta$ es una función continua y positiva.\\
la \textbf{topología $C^\infty$ de Whitney} tiene como base la unión de todos los abiertos de las topologías $C^k$.

\begin{obs}
	La topologia de Whitney es Baire no metrizable.\\
	El espacio de jets entre dos variedades en un fibrado sobre cada variedad. Esto define la estructura suave de $J^k(n,p)$, la topología coincide con la euclidiana... aunque eso no importa.
\end{obs}


\begin{pro}
	\begin{enumerate}
		\item Considerando la topologia de Whitney $j^k : C^\infty(n,p) \rightarrow C^\infty (\mathbb{R}^n,  J^k (n,p)) $ es continua.
		\item $\phi_*: C^\infty (n,p) \rightarrow (n,q)$, con $\phi : \mathbb{R}^p \rightarrow \mathbb{R}^q$ (covariante) es continua.
		\item Si $\phi \in \operatorname{Dif} (\mathbb{R}^n)$ y $\psi \in C^\infty (n,n)$ entonces $f \mapsto \psi \circ f \circ \phi^{-1}$ es continua. Si $\psi$ es un difeomorfismo, entonces es un homeomorfismo.
	\end{enumerate}
\end{pro}




\begin{teo}[Transversalidad de Thom]
	Sean $Q_1 \cdots Q_p$ subvariedades suaves de $J^k(n,p)$. Entonces el conjunto de las aplicaciones $f$ en $C^\infty(n,p)$ tales que $j^k (f)$ es transversal a todos los $Q_i$ es denso.
\end{teo}



Un mapa $f \in C^\infty (n,p)$ es \textbf{globalmente estable} si existe una vecindad de $f$ contenida en una única $\dif n \times \dif p-$órbita. En realidad, dicha órbita será abierta.

\begin{teo}
$f$ propio. Son equivalentes.
\begin{enumerate}
\item globalmente estable.
\item Para todo $p $, $S \subset f^{-1}(p)$, $f: ( , S) \rightarrow (, p)$ es estable.
\end{enumerate}
\end{teo}

%Mather toerema


\subsection{Singularidades de primer orden}

Sea $f: \mathbb{R}^n\rightarrow \Re ^p $, definimos el conjunto de singularidades de primer orden $$ \Sigma ^i (f) =\{ x \in \Re ^n \; |\; \dim \ker df (x) =i\}. $$

También definimos $$ \Sigma ^i = \{  \sigma \in J^1 (n,p) \; | \; \dim \ker j^1 \sigma = i \} $$
Note que $\Sigma ^i (f) = j^{-1} (f) (\Sigma ^ i ) $, por tanto es una subvariedad.\\

\begin{pro}
$\cod \Sigma ^i = i (p-n+i)$ en $J^1(n,p)$. 
\end{pro}

\dem aplicar lema de álgebra lineal y luego transversalidad de Thom.\\

\begin{cor}
El conjunto de aplicaciones con $1-$jet transversal a todos los $\sigma i$ es denso, para ese conjunto $\cod \sigma ^i (f) = i (|p-n|+i)$
\end{cor}




\subsection{Singularidades de Thom-Boardman}

Definir símobolo de Boardman







\section{Gérmenes simples}

Una singularidad finitamente determinada $f\in \mm \mathcal{O}_{n,p}$ es \textbf{simple} si para todo $k$ grande existe un vecindad $V$ de $j^k(f)$ en $J^k (n,p)$ que contiene un número finito de órbitas de la acción de $ J^k \mathcal{G} $ en $ J^k (n,p)$.\\
También se puede definir usando deformaciones y la topología del espacio de funciones.
\begin{lem}
es suficiente tomar una deformación versal
\end{lem}
Gérmenes no simples... comentarios.... ver en las notas... también caso Arnold

\begin{pro}
A simple equivale K simple, equivale R simples
\end{pro}
 %no encuentor la demostración






\section{Algunos gérmenes estables}

Vamos a clasificar gérmenes del plano en el plano.

\begin{teo}[Whitney]
Las únicas singularidades aisladas que aparecen en los discriminantes de una aplicación estable del plano en el plano son cúspides y dobleces transversales.
\end{teo}





\chapter{Falso epílogo}

Determinación finita.\\
Variedades.



\part{Topología de las singularidades}

\chapter{La fibración de Milnor}

\begin{teo}
Existe un vecindad puntuada del origen tal que $f$ es una fibración, con fibra homotópica al bouquet de esferas (de codimensión 1 por regularidad), de cantidad kgula al número de milnor.
\end{teo}
	
\chapter{Cálculo número de Milnor y  Tjurina}
Poliedro de Newton	

\begin{pro}

\end{pro}

	
\chapter{Estratificaciones}

cómo varían los estratos con deformaciones (poliedro de Newton)

	
	
	
\part{Casos particulares y generalizaciones}	
	
\chapter{Intersección completa}	




\section{Singularidades de esquemas afines}

\section{Cohen Macaulay}

\section{Gorenstein}


\chapter{Variedades determinantales}
	
\end{document}




\chapter{Determinación finita}

El concepto de determinación finita es interesante por sí mismo, además está relacionada con la codimensión finita.


Definición $\mathcal{G}-$finito. 


\section{Criterios}

\begin{teo}[Codimensión y determinación finita]
	Sea $f: (\kk ^n, 0 ) \rightarrow (\mathbb{K}^p,0)$. Son equivalentes:
	\begin{enumerate}
		\item $f$ es $\mathcal{G}$-finito.
		\item existe $r$ tal que $\mathfrak{m}^{r+1} \theta _f$, 
		\item $\mathcal{G}-\operatorname{cod}(f) < \infty$
		\item $\mathcal{G}_e-\operatorname{cod}(f) < \infty$.
	\end{enumerate}
\end{teo}

\begin{teo}[específico]
	\begin{enumerate}
		\item Si $f$ es $r-\mathcal{G}-$finito, entonces $\mathfrak{m}_n ^{r+1} \subset L \mathcal{G} \cdot f $.
		\item Para $\mathcal{G}= \mathcal{L}, \mathcal{A}$, si $ \mathfrak{m}_n ^{r+1}$ entonces $f$ es $(2r+1)-\mathcal{G}-$finito.
		Para $\mathcal{G}= \mathcal{L}, \mathcal{A}$, si $ \mathfrak{m}_n ^{r+1}$ entonces $f$ es $(r+1)-\mathcal{G}-$finito.
	\end{enumerate}
\end{teo}

%usamos un resultado de comutaiva que relaciona codimiensión con completamento


\section{Caso particular $\rg$}

\begin{pro}[Lema de Thom-Levine]
	
\end{pro}



%cobordismo
%ya debería saber toería de morse 

