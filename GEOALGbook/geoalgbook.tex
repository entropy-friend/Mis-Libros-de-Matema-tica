\documentclass[12pt]{book}
\usepackage[utf8]{inputenc}
\usepackage[spanish]{babel}
\usepackage[a4paper, total={7in, 10in}]{geometry}
\usepackage{tgbonum}
\usepackage{amsmath}
\usepackage{amssymb}
\usepackage{graphicx}
\newtheorem{eje}{Ejemplo}
\newtheorem{teo}{Teorema}
\newtheorem{obs}{Observación}
\newtheorem{lem}{Lema}
\newtheorem{pro}{Proposición}
\newtheorem{ex}{Ejercicio}
\newtheorem{cor}{Corolario}

\newcommand{\z}{\mathbb{Z}}
\newcommand{\n}{\mathbb{N}}
\newcommand{\re}{\mathbb{R}}
\begin{document}
\title{Geometría Algebraica}	
\author{Abraham Rojas}
\date{ }

\maketitle
\tableofcontents
\part{Parte I}

\chapter{Introducción}
%estructura del libro 

Este es un libro de Geometría. \\

Asumiremos conocimientos de Álgebra Conmutativa, aunque hay un resu,ent en el apéndice.

Me gusta distinguir entre dos conceptos: la Geometría Algebraica, el estudio de la geometría utilizando espacios algebraicos y herramientas algebraicas, y por otro, lo que me gusta llamar "Álgebra Geométrica", o sea, utilizar herramientas geométricas para estudiar estructuras algebraicas. Nosotros seguiremos el primer enfoque.\\

Por otro lado, podemos estudiar la Geometría utilizando diferentes tipos de espacios ambiente y diversas herramientas. \\

Obviamente, toda distinción que se haga entre áreas del conocimiento no deja de ser artificial. \\

la introducción de un libro me parece una buena oportunidad para comentar temas qeu no serán explicados en el mismo. POr ejemplo, Álgebra Geometrica


\chapter{Guardando información geométrica}



%en realidad, esto debería ir en el libro de goem. diferencial
%
%$M$ espacio topológico. \textbf{Fibrado vectorial} de rango $k$ sobre $M$ es un espacio topológico $E$ junto con un mapa continuo sobreyectivo $\pi : E \rightarrow M$ tal que :
%\begin{enumerate}
%\item Para cada $p \in M$, la fibra $E_p=\pi^{-1}(p)$ sobre $p$ tiene la estructura de un espacio vectorial real de dimensión $k$.
%\item Para cada $p \in M$, existe un vecindad $U$ de $p$ en $M$ y un homeomorfismo $\Phi: \pi^{-1}(U) \rightarrow U \times \mathbb{R}^k$ (\textbf{trivialización}), tal que
%\begin{itemize}
%\item $\pi_U \circ \Phi=\pi$ ($\pi_U $ es la proyección sobre $U$);
%\item la restricción de $\Phi$ a una fibra $E_q$ es un isomorfismos de espacios vectoriales entre $E_q$ y $\{q\} \times \mathbb{R}^k \cong \mathbb{R}^k$.
%\end{itemize}
%\end{enumerate}
%
%si $M$ son variedades suaves (con o sin borde), $/pi$ es un mapa suave y las trivizalizaciones locales son difeomorfismos, entonces $E$ es un \textbf{fibrado vectorial suave}




\section{Haces}

Podemos pensar en los haces como generalización del concepto de fibrado.\\

Sea $X$ un espacio topológico.\\

Consideremos la categoría $\mathfrak{U}$, cuyos elementos son los abiertos de $X$ y cada morfismo $U\rightarrow V$ proviene de la inclusión $U \subset V$.\\
Un \textbf{pre-haz} (de grupos abelianos) sobre $X$ es un funtor contravariante $$ \mathcal{F}:\mathfrak{U} \rightarrow \mathfrak{Sets}$$
Dados $U \subset $ en $\mathfrak{U}$, el \textbf{morfismo de restricción} es el morfismo correspondiente a la inclusión. Dado $s \in \mathcal{F}(V)$, denotaremos su \textbf{restricción} en $U$ (i.e., su imagen por el morfismo de restricción) por $s|_U$. A veces, denotaremos $\Gamma(U, \mathcal{F})= \mathcal{F}(U)$.\\
Un \textbf{haz} (de grupos abelianos) es un pre-haz que satisface
\begin{itemize}
\item (Determinación) Sean $U$ un abierto, $\{V_i\}_{i\in I}$ un cubrimiento abierto y $s\in U$ tal que $s|_{V_i}=0$ para todo $i$, entonces $s=0$.
\item (Construcción) Sean $U$ un abierto, $\{V_i\}_{i\in I}$ un cubrimiento abierto y elementos $s_i \in \mathcal{F}(V_i)$ para cada $i$, tal que $s_i |_{V_i \cap V_j} = s_j |_{V_i \cap V_j}  $ para todo $i,j$. Entonces existe $s\in \mathcal{F}(U)$ tal que $ s|_{V_i} = s_i$ para cada $i$ 
\end{itemize}


\begin{eje}[Variedades suaves]

\end{eje}

\begin{eje}[los clásicos]
	contenidos...
\end{eje}

\subsection{gavijas}


\subsection{Hacificación}

%debería ser después de álgebra conmutativa 2 
%\section{Cohomologia de haces}
%
%
%
%\subsection{Cohomología por resoluciones}
%
%
%
%\subsection{Cohomología de Čech}



\subsection{Más ejemplos y motivación}

%solo enunciar resultados, no dar ideas y no demostraciones





\section{Divisores en variedades complejas}

%\subsection{Teoría topológica de intersección}







\chapter{La Geometría de los anillos conmutativos}


%problemas clásicos: dimensión (teoría de eliminación, espacio tangente, intersecciones)\\
%dos problemas son equivalentes si las variedades son isomorfas.
En adelante, $A$ será un anillo conmutativo con unidad. \\

\section{Geometría Algebraica Clásica}

Sea $K$ un cuerpo algebraicamente cerrado, el \textbf{espacio afín de dimensión $n$} sobre $K$ es definido como $K^n$, denotado por $ \mathbb{A}^n _K$ o $\mathbb{A}^n$.\\


Sea $S \subset K[X_1, \ldots , X_n]$, definimos el \textbf{conjunto algebraico} (asociado a $S$) como $$Z (S) =\{ P \in \mathbb{A} ^n \; | \; f(P) =0, \; \forall f \in S  \}. $$ 

Es obvio que si $S \subset T \subset K[X_1, \ldots , X_n]$ entonces $Z (J) \subset Z (I)$. 

\begin{eje}
	\item conjuntos unitarios son llamados \textbf{puntos}, estos son conjuntos algebraicos, también cantidad finita de puntos.
	\item recta afín, curvas algebraicas planas.
	\item Note que $Z(f)^n = Z(f)$, luego si $I $ es un ideal, entonces $Z(\sqrt{I})= Z(I)$. 
\end{eje}

%algunos preguntas interesantes 


Dado $h\in K[X_1, \ldots , X_n]$, definimos $D(h)= \mathbb{A}^n \setminus Z(f) $.

\begin{pro}[Topología de Zariski ingenua]
	Sea $\{I_\lambda\}_{\lambda\in \Lambda}$ una familia de ideales de $K[X_1, \ldots , X_n]$, tenemos que 
	\begin{equation*}
		\bigcap _{\lambda \in \Lambda } Z (J _ \lambda) = Z \left( \sum _{\lambda \in \Lambda} j _\lambda \right).\\
		\bigcup _ {i=1} ^k Z(J _ {\lambda_i}) = Z \left( \bigcap_{i=1 }^k J _{\lambda_i}  \right) =  Z \left( \prod_{i=1 }^k J _{\lambda_i}  \right)
	\end{equation*}
	
	Así, los conjuntos algebraicos forman la familía de conjuntos cerrados correspondiente a una topología en $\mathbb{A}^n$, llamada \textbf{topología de Zariski}. Los conjuntos $D(h)$ forman una base para esta topología.
\end{pro}



Sea $X \subset \mathbb{A}^n $, definimos el \textbf{ideal asociado} a $X$ como $$I(X) = \{ f \in K[X_1, \ldots , X_n]\; | \; f(P) =0, \; \forall P \in X  \}.$$

Es claro que si $X \subset Y \subset \mathbb{A}^n$ entonces $I(X) = I(Y)$. Además, si $f^n\in I(X)$ entonces $f\in I(X)$, luego $I(X)$ es un ideal radical. \\
Por otro lado, el Teorema de la base de Hilbert implica que $I(X)$ siempre es finitamente generado. 


\begin{teo}[Nullstellensatz geométrico]
	Sea $K$ un cuerpo algebraicamente cerrado. Tenemos que $$\sqrt{I} = I(Z (I))$$
\end{teo}

\begin{cor}
	Existe una biyección entre el con
\end{cor}


%variedades proyectivas




\subsection{Más que una equivalencia entre categoría}








\section{Espectros}


El \textbf{espectro} de $A$ es el conjunto de sus ideales primos, denotado por $\operatorname{Spec}(A)$. Podemos dotar a este conjunto de la \textbf{Topología de Zariski}, donde los conjuntos cerrados son de la forma $$ V(I) := \{ \mathfrak{p} \in \operatorname{Spec} A \;| \; I \subset \mathfrak{p} \}, \mbox{ donde } I \mbox{ es un ideal}.$$

\subsection{Topología de Zariski}

%axiomas de topologia


Dado um elemento $h\in A$, definimos $$D(h)= \{\mathfrak{p} \; |\; h \notin\mathfrak{p}\}$$



Si $\phi : A \rightarrow B$ es un morfismo de anillos, definimos el \textbf{mapa induzido} por $$ \operatorname{Spec}(\phi) : \operatorname{Spec} (B) \rightarrow \operatorname{Spec}(A)$$

\begin{teo}
\begin{enumerate}
\item $\{D(H)\}_{h\in A}$ es una base de abierto de la topología de Zariski.
\item Los mapas inducidos por morfismos de anillos son continuos.
\item $\operatorname{cl} \{\mathfrak{p}\} = V(\mathfrak{p})$,  en particular, un ideal primo es un punto cerrado si y solo si este es maximal.
\item $\operatorname{Spec}(A)$ es compacto. 
\item Sea $I$ un ideal, el morfismo inducido de la proyección natural $A \rightarrow A/I$ es un homeomorfismo entre $\operatorname{Spec}(A/I) $ con $V(I)$.

\end{enumerate}
\end{teo}

\begin{eje}

\end{eje}

%definición de dimensión topológica, irreducibles, espacio topológico noetheriano, punto genérico 

%existe una biyección entre los puntos cerrados de spec y 

\begin{teo}[Nullstellensatz]
Sea $ \sqrt{I}= I(Z(I))$
\end{teo}

%problemas de la topología de Zariski




\section{Esquemas}



\subsection{Productos fibrados}



\chapter{Morfismos proyectivos}

El \textbf{espacio proyectivo} de dimensión $n$ sobre $k$, denotado por $\mathcal{P}^n_k$, está dado por $\mathbb{A}^{n+1}_k /\sim$, dada por la relación de equivalencia $$
\left(a_0, \ldots, a_n\right) \sim\left(\lambda a_0, \ldots, \lambda a_n\right) \text { for all } \lambda \in k, \lambda \neq 0 \text {. }
$$


Sabemos que $S= k[x_0 , \ldots, x_n]$ es un anillo, siendo $S_d$ las combniaciones lineales de monomios de grado $d$.\\
Sea $T$ un conjunto de polinomios homogéneos, el \textbf{conjunto de zeros} de $T$ es definido como
$$
Z(T)=\left\{P \in \mathbf{P}^n \mid f(P)=0 \text { for all } f \in T\right\} .
$$
si $I$ es un ideal homogéneo, definimos $Z(I)$ como el conjunto de ceros del conjunto de elementos homogéneos de $I$.\\
$Y \subset \mathbb{P}$ es \textbf{conjunto algebraico proyectivo} si existe un conjunto de polinomios homogéneos $T$ tal que $Y= Z(T)$.\\
Sea $Y\subset \mathbb{P}^n$, definimos su \textbf{ideal homogéneo} como $$I(Y)= \langle \{ f\in S \, | \, f \mbox{ es un polinomio homogéneo y } f(P)=0 \mbox{ para todo } P \in Y \}  \rangle $$ 
Si $Y$ es un conjunto algebraico, su \textbf{anillo de coordenadas homogéneas} está definido como $S(Y)= S/I(Y)$.

La topologia de Zariski en $\mathbb{P}^n$ es aquella cuyos cerrados son los conjuntos algebraicos proyectivos.\\
Una \textbf{variedad proyectiva} es un conjunto algebraico irreducible en $\mathbb{P}^n$, con la topología inducida. Un conjunto abierto de una variedad proyectiva es un \textbf{variedad casi-proyectiva}.


















\part{Teoría}


\chapter{Clases Características}

\section{Divisores, encajes y diferenciales}

\subsection{Morfismos proyectivos}

\subsection{Teorema de Riemann-Roch}

\subsection{Dualidad de Serre}

\section{Variedades Jacobianas}







\chapter{Algunos casos particulares}

\section{Geometría birracional}

\section{Curvas}

\section{Superficies}

\section{Geometría Compleja}

\section{Geometría Diofantina}












\chapter{Deformaciones}

\section{Teoría de Intersección}


Me llevó mucho tiempo tratar de entender los cómos y los porqués de las definiciones. Valió mucho la pena, pero creo que un\\
Lo más difícil es dar un concepto (y en particular, una definición) precisa de los número de intersección y grado de una variedad.
Así que comenzaremos dando algunas condiciones (que pueden tomarse como axiomas) que cumplen los número de intersección\\

número de intersección lo que implica definición de grado

El primer paso percibir que el número de intersección varía continuamente con la posición de las variedades. Considere el siguiente

\begin{eje}
	%x n  y ^m tiene número de intersección nm
\end{eje}

Ya podemos ver una de las posibles definiciones del número de intersección


\subsection{Resultantes}

Sea $A$ un dominio integral. Sean $P(x)= \sum^p_{k=0} a_{p-k} x^k$ y $Q(x)= \sum^q_{k=0} b_{q-k} x^k$ polinomios homogéneos en $A[x]$, de grados $p$ y $q$ respectivamente. El \textbf{resultante} de $P$ y $Q$ es el determinante de la función $A-$lineal $A[x]_{\leq p} \times A[x]_{\leq q} \rightarrow A[x]_{\leq p+q}$ definida por $$ (A,B) \rightarrow Ap+BQ $$

El resultante posee muchas propiedades interesantes para el álgebra computacional. Para nuestros fines, será necesaria la siguiente 

\begin{pro}
\begin{enumerate}
\item El resultante se anula si y solo si los polinomios tienen una raíz común en cuerpo algebraicamente cerrado que contiene al anillo de coeficientes.
\item es un polinomio homogéneo, de grado $p$ respecto a los $a_i$ y de grado $q$ respecto a los $b_j$.

\end{enumerate}
\end{pro}

\chapter{Cohomologia Ètale}



%resultante y U-resultante


\chapter{Mirror Symmetry}







\appendix

\chapter{Un poco de Análisis}

\chapter{Un recuento de Álgebra Conmutativa}

\chapter{Álgebra Homológica}

\section{Categorías Abelianas}






\end{document}





\section{Topología algebraica}

\subsection{Recubrimientos y grupo fundamental}

\subsection{Homologías y Cohomologías}







\section{Otros}

\subsection{Estructuras exóticas}

\subsection{Grupos de Lie}

\subsection{Cocientes}

\subsection{Teoría de Galois Diferencial}